\section{Software and Documentation}
\label{sec:software}

The functions specified in this API shall be implemented exactly as
defined in the \lib. The header file used in the test driver is
provided on \theprojectwebsite and as part of the validation package.

\subsection{Restrictions}
\label{subsec:software-restrictions}
Participants shall provide binary code in the form of a \lib only
(i.e., no source code or headers). \Libs must be submitted in the form of a
dynamic/shared library file (i.e., ``.so'' file). This library shall be known as
the ``core'' library, and shall be named according to the guidelines in
\Cref{subsec:software-naming}. Static libraries (i.e., ``.a'' files) are not
allowed. Multiple shared libraries are permitted if technically required and are
compatible with the validation package. Any required libraries that are not
standard to \os or installed on the testing hardware must be built and submitted
alongside the ``core'' library. All submitted \libs will be placed in a single
directory, and this directory will be added to the runtime library search path
list.

Individual \libs provided must not include multiple modes of operation or
algorithm variations. No switches or options will be tolerated within one
library. Access to a configuration directory of configuration files is provided,
and if necessary, such algorithm configuration changes would occur during the
call to an appropriate \code{init} method.

The \lib shall not make use of threading, forking, or any other multiprocessing
techniques, \textit{except during calls to \code{finalizeEnrollment()} and
\code{init} methods}. The \testdriver operates as a Message Passing Interface
(MPI) job over multiple compute nodes, and then forks itself into many
processes. In the test environment, there is no advantage to threading. It
limits the usefulness of NIST's batch processing and makes it impossible to
compare timing statistics across participants. During
\code{finalizeEnrollment()} and \code{init} methods, up to $C-1$ threads may be
used for threading. This value should be queried at runtime using
\code{std::thread::hardware_concurrency}.

The \lib shall remain stateless. It shall not acknowledge the existence of other
processes running on the test hardware, such as through \code{semaphore}s or
\code{pipe}s. It shall not acknowledge the existance of other machines on the
network. It shall not write to arbitrary file system locations. The only file
system writes should occur during calls to API methods requiring file system
writes. It shall not attempt network requests.

\subsection{External Dependencies}
It is preferred that the API specified by this document be implemented in a
single ``core'' library. Additional libraries may be submitted that support this
``core'' library file (i.e., the ``core'' library file may have dependencies
implemented in other libraries if a single library is not feasible).

Use of compiler and architecture optimizations is allowed and strongly
encouraged, but \libs must be able to run on the following processor types:
\begin{itemize}
	\item AMD Opteron 8376HE
	\item Intel Xeon E5405, X5680, X5690, X7560
	\item Intel Xeon E5-2680
\end{itemize}

\subsection{Naming}
\label{subsec:software-naming}
The ``core'' \lib submitted shall be named in a predefined format. The first
part of the \lib's name shall be \code{libN2N_}. The second piece of the \lib's
name shall be a case-sensitive unique identifier that has been provided to you
by the test administrator, followed by an underscore. The final part of the
\lib's name shall be a version number, followed by the file extension
\code{.so}. Supplemental libraries may have any name, but the ``core'' library
must be dependent on supplemental libraries in order to be linked correctly. The
\textbf{only} library that will be explicitly linked to the \testdriver is the
``core'' library, as demonstrated in \Cref{subsec:software-environment}.

The identifier and version number provided in the file name must match the
values output by \code{getIDs()}. The version number must be incremented with
every subsequent submission. With this naming scheme, \textbf{every ``core''
library received shall have a unique filename}. Incorrectly named or versioned
\libs will be rejected.

\textbf{Example}\\
Company X requests to participant in this software evaluation by submitting
their public signing and encryption key. The project sponsor replies with the
identifier ``01F3'' for use in Company X's \implementation. Company X submits a
\lib named \code{libN2N_01F3_1.so} with revision 1 of their algorithm. This
library assigns \code{01F3} to the \code{identifier} out parameter and
\code{1} to the \code{revision} out parameter of  \code{getIDs()}. It is
determined that Company X's validation fails and the \lib is rejected. Company
X submits revision 2 to correct the defects in 1. Company X updates the value
of \code{revision} in their implementation of \code{getIDs()} to \code{2} and
renames their library to \code{libN2N_01F3_2.so}.

\subsection{Operating Environment}
\label{subsec:software-environment}

The \lib will be tested in non-interactive ``batch'' mode (i.e., without
terminal support) in an isolated environment (i.e., no Internet or other network
connectivity). Thus, the \lib shall not use any interactive functions, such as
graphical user interface calls, or any other calls that require terminal
interaction (e.g., writes to \code{stdout}).

NIST will link the provided library files to a \cpp language test driver
application using the compiler \code{g++} (version \code{4.8.5-4}, via
\code{mpicxx}) under \textbf{\os}. For example:
\begin{center}
	\code{mpicxx -std=c++11 -o N2N N2N.cpp -Llib -lN2N_01FE_2}
\end{center}

Participants are required to provide their \libs in a format that is linkable
using \code{g++} with the \testdriver. All compilation and testing will be
performed on 64-bit hardware running \os. Participants are strongly encouraged
to verify library-level compatibility with \code{g++} on \os \textbf{prior to}
submitting their software to avoid unexpected problems.

\subsection{Usage}
\label{subsec:software-usage}

The \lib shall be executable on any number of machines without requiring
additional machine-specific license control procedures, activation, hardware
dongles, or any other form of rights management.

The \lib usage shall be \textbf{unlimited}. No usage controls or limits based
on licenses, execution date/time, number of executions, etc., shall be enforced
by the \lib.

\subsection{Validation and Submitting}
\label{subsec:software-validation}

A \textit{validation package} that will link the participant's ``core''
\lib to a sample \testdriver will be provided. Once the validation successfully
completes on the participant's system, a file with validation data and the
participant's \lib will be created. After being signed and encrypted,
\textbf{only} this signed and encrypted file shall be submitted
to \emailaddress. Any \lib submissions not generated by an unmodified copy of
the \project validation package will be rejected. Any \lib submissions that
generate errors while running the validation package on the test hardware will
be rejected. Validation packages that have recorded errors while running on the
participant's system will be rejected. Any \lib submissions not generated with
the current version of the validation package will be rejected. Any submissions
of successful validation runs not created on \os will be rejected.

This validation step is crucial to determine that the \lib submitted by
participants is operating correctly and as expected in different hardware
environments. Any difference in candidate lists or similarity scores on the test
hardware as compared to the participant's hardware, regardless of the number of
nodes used during the first stage of identification, will not be tolerated.
Differences in search and enrollment templates on the test hardware and
participant hardware are permitted, but strongly discouraged.

\subsection{Speed}
\label{subsec:software-speed}

Timing tests will be run and reported on a sample of the ``legacy'' dataset
prior to completing the entire test. Submissions that do not meet the timing
requirements on the dataset sample will be rejected. Participants may resubmit a
faster submission immediately.

\begin{table}
	\centering
	\rowcolors{2}{white}{gray!10}
	\begin{tabular}{>{\centering}m{2.4in}>{\centering}m{1.1in}c}
		\toprule
		\rowcolor{white}
		\textbf{Method} & \textbf{Maximum Time} &
		    \textbf{Measurement} \\
		\midrule
		\code{getIDs()} & $0$ seconds & Exact \\
		\code{initMakeEnrollmentTemplate()} & $5$ minutes & Exact \\
		\code{makeEnrollmentTemplate()} & $3$ seconds & $90^{th}$
		    percentile \\
		\code{finalizeEnrollment()} & $90$ minutes per $1$ million
		    subjects & Exact\\
		\code{initMakeSearchTemplate()} & $5$ minutes & Exact \\
		\code{makeSearchTemplate()} & $3$ seconds & $90^{th}$
		    percentile \\
		\code{initIdentificationStageOne()} & $5$ minutes & Exact \\
		\code{initIdentificationStageTwo()} & $5$ minutes & Exact \\
		\code{identifyTemplateStageOne()} $+$
		    \code{identifyTemplateStageTwo()} & $1$ minute &
		    $90^{th}$ percentile \\
		\bottomrule
	\end{tabular}
	\captionsetup{font=footnotesize}
	\caption{Speed requirements for API methods.}
	\label{tbl:software-speed}
\end{table}

