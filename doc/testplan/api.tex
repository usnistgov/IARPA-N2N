\section{API Description}

The \project software API is broken into three distinct parts:
\begin{itemize}
	\item Feature Extraction
		\begin{itemize}
			\item Enrollment Templates
			\item Search Templates
		\end{itemize}
	\item Enrollment Set Generation
	\item Identification
		\begin{itemize}
			\item Stage One (partial search)
			\item Stage Two (complete search)
		\end{itemize}
\end{itemize}

The process starts with creating many enrollment and search templates. Next, all
enrollment templates are combined by the participant into a proprietary
enrollment set, suitable for searching. Finally, search templates are searched
against the enrollment set.

It is expected that the total memory needed to store the entire enrollment set
in Random Access Memory (RAM) will exceed the RAM capacity of a single test
hardware node. For this reason, identification is broken into two stages.
During creation of the enrollment set, the \implementation is provided the
number of nodes that will be available during the first stage of
identification. The \implementation should use this information to partition
the enrollment set into that many parts.  Identification calls during the first
stage will be provided a number, indicating the enrollment set partition in
which to search. The \testdriver will ensure that each search template will be
searched against all partitions. During \stagetwoidentification, all
information generated from all the partition searches during
\stageoneidentification will be provided to the \implementation simultaneously.
This information should be processed and formed into a candidate list returned
to the caller.

\subsection{Enrollment Template Creation}
\label{subsec:api-enrollment_template}

The \lib will be responsible for creating several million enrollment templates.
First, the \testdriver provides an opportunity for the \lib to load any
configuration files and update the state of the implementation object returned.
Next, the \testdriver \code{fork}s and repeatedly calls
\code{makeEnrollmentTemplate()}. This method provides the \lib with a variable
number of standard images or proprietary \scanner data, all from a single
subject. The \lib should provide a single enrollment template representing the
data of the passed-in subject.

A Unified Modeling Language (UML) sequence diagram for this process is provided
in \Cref{fig:api-enrollment_template}.

\subsubsection{Failure to Extract}
Failures to extract shall be reported by the \lib returning
\code{ReturnStatus::FailedToExtract} and setting an appropriate value for the
enrollment template (likely a 0-byte buffer). All enrollment templates,
regardless of the value of \code{ReturnStatus}, will be provided to
the enrollment set generation step described in \Cref{subsec:api-finalization}.

\begin{figure}
	\centering
	\begin{sequencediagram}
		\newthread{driver}{\TestDriver}
		\newinst{driverchild}{\shortstack{\TestDriver\\Child}}
		\newinst{logger}{Logger}
		\newinst{datastore}{Storage}
		\newinst{interface}{\code{N2N::Interface}}
		\newinst{impl}{\shortstack{\code{N2N::Interface}\\
		    Implementation}}

		\begin{sdblock}{Enrollment Template Creation}{}
			\begin{call}{driver}{\code{getImplementation()}}
			    {interface}{\code{std::shared_ptr<N2N::Interface>}}

				\begin{call}{interface}{new}{impl}{\code{this}}
				\end{call}

			\end{call}


			\begin{call}{driver}{\code{getIDs()}}{impl}
			    {E-mail, Revision, ID, \code{ReturnStatus}}
			\end{call}

			\begin{call}{driver}{E-mail, Revision, ID}{logger}{}
			\end{call}

			\begin{sdblock}{Initialization}
			    {\textit{Participant multi-threading is permitted}}

				\begin{call}{driver}
				    {\code{initMakeEnrollmentTemplate(...)}}
				    {impl}{\code{ReturnStatus}}
				\end{call}

				\begin{call}{driver}{\code{ReturnStatus}}
				    {logger}{}
				\end{call}
			\end{sdblock}

			\begin{call}{driver}{new}{driverchild}{exit}
				\begin{sdblock}{Feature Extraction Loop}
				    {\textit{$[1,n]$ processes, no
				    participant multi-threading}}
					\begin{call}{driverchild}
					{\code{makeEnrollmentTemplate(...)}}
					{impl}{\shortstack{Enrollment
					Template\\\code{ReturnStatus}}}
						\begin{callself}{impl}
						    {\shortstack{Extract\\
						    Features}}{}
							\postlevel
						\end{callself}
					\end{call}

					\begin{call}{driverchild}
					    {Enrollment Template}{datastore}{}
					\end{call}
					\postlevel

					\begin{call}{driverchild}
					    {\shortstack{Duration\\
					    \code{ReturnStatus}}}{logger}{}
					\end{call}
				\end{sdblock}
			\end{call}


		\end{sdblock}
	\end{sequencediagram}

	\captionsetup{font=footnotesize}
	\caption{Enrollment Template Creation.}
	\label{fig:api-enrollment_template}
\end{figure}

\subsection{Search Template Creation}
\label{subsec:api-search_template}

The process of creating search templates is identical to the creation of
enrollment templates(\Cref{subsec:api-enrollment_template}). One significant
change is that the \lib will be told during the initialization step whether or
not latent imagery will be provided as sources for feature extraction. It is
expected that participants will need to load a different algorithm or set of
configuration files to process latent imagery. Only the type of imagery
specified during initialization will be provided during the
\code{N2N::Interface} implementation's lifetime.

A UML sequence diagram for this process is provided in
\Cref{fig:api-search_template}.

\subsubsection{Failure to Extract}
Failures to extract shall be reported by the \lib returning
\code{ReturnStatus::FailedToExtract} and returning a 0-byte buffer for the
search template. These search templates will not be passed to the two-stage
identification methods.

\begin{figure}
	\centering
	\begin{sequencediagram}
		\newthread{driver}{\TestDriver}
		\newinst{driverchild}{\shortstack{\TestDriver\\Child}}
		\newinst{logger}{Logger}
		\newinst{datastore}{Storage}
		\newinst{interface}{\code{N2N::Interface}}
		\newinst{impl}{\shortstack{\code{N2N::Interface}\\
		    Implementation}}


		\begin{sdblock}{Search Template Creation}{}
			\begin{call}{driver}{\code{getImplementation()}}
			    {interface}{\code{std::shared_ptr<N2N::Interface>}}
				\begin{call}{interface}{new}{impl}{\code{this}}
				\end{call}
			\end{call}


			\begin{call}{driver}{\code{getIDs()}}{impl}{E-mail,
			    Revision, ID, \code{ReturnStatus}}
			\end{call}

			\begin{call}{driver}{E-mail, Revision, ID}{logger}{}
			\end{call}

			\begin{sdblock}{Initialization}{\textit{Participant
			    multi-threading is permitted}}
				\begin{call}{driver}
				    {\code{initMakeSearchTemplate(...)}}{impl}
				    {\code{ReturnStatus}}
				\end{call}

				\begin{call}{driver}{\code{ReturnStatus}}
				    {logger}{}
				\end{call}
			\end{sdblock}

			\begin{call}{driver}{new}{driverchild}{exit}
				\begin{sdblock}{Feature Extraction Loop}
				    {\textit{$[1,n]$ processes, no participant
				    multi-threading}}
					\begin{call}{driverchild}
					    {\code{makeSearchTemplate(...)}}
					    {impl}{\shortstack{Search
					    Template\\\code{ReturnStatus}}}
						\begin{callself}{impl}
						    {\shortstack{Search\\
						    Features}}{}\postlevel
						\end{callself}
					\end{call}

					\begin{call}{driverchild}
					    {Search Template}{datastore}{}
					\end{call}

					\postlevel

					\begin{call}{driverchild}
					    {\shortstack{Duration\\
					    \code{ReturnStatus}}}{logger}{}
					\end{call}
				\end{sdblock}
			\end{call}


		\end{sdblock}
	\end{sequencediagram}

	\captionsetup{font=footnotesize}
	\caption{Search Template Creation.}
	\label{fig:api-search_template}
\end{figure}

\subsection{Enrollment Set Creation}
\label{subsec:api-finalization}

Once all enrollment templates have been created, the \testdriver will provide
all enrollment templates to the \lib for processing into an enrollment set. The
file format of the enrollment set is not specified, but please note that the
file system running on the test hardware performs poorly with many small
files. Some sort of ``database'' flat-file format (e.g., BerkeleyDB, SQLite,
etc.) is recommended.

The number of test hardware nodes used for \stageoneidentification
will be provided to the \implementation during this process. If necessary for
your implementation of identification, ensure that the enrollment set is
partitioned into at least this many partitions.

Participants must make a copy of all enrollment templates provided, as it will
be their last opportunity to have access to the data in this form.

A UML sequence diagram for this process is provided in
\Cref{fig:api-finalization}.

\begin{figure}
	\centering
	\begin{sequencediagram}
		\newthread{driver}{\TestDriver}
		\newinst{logger}{Logger}
		\newinst{datastore}{Storage}
		\newinst{interface}{\code{N2N::Interface}}
		\newinst{impl}{\shortstack{\code{N2N::Interface}\\
		    Implementation}}


		\begin{sdblock}{Finalization of Enrollment Set}{}
			\begin{call}{driver}{\code{getImplementation()}}
			    {interface}{\code{std::shared_ptr<N2N::Interface>}}
				\begin{call}{interface}{new}{impl}{\code{this}}
				\end{call}
			\end{call}

			\begin{call}{driver}{\code{getIDs()}}{impl}{E-mail,
			    Revision, ID, \code{ReturnStatus}}
			\end{call}

			\begin{call}{driver}{E-mail, Revision, ID}{logger}{}
			\end{call}

			\begin{sdblock}{Finalization}{\textit{Participant
			    multi-threading is permitted}}
				\begin{call}{driver}
				    {\code{finalizeEnrollmentSet(...)}}{impl}
				    {\code{ReturnStatus}}
					\begin{call}{impl}{Read Templates}
					    {datastore}{\shortstack{Enrollment\\
					    Templates}}
						\postlevel
					\end{call}

					\begin{callself}{impl}
					    {\shortstack{Process\\Templates}}{}
					\end{callself}

					\begin{call}{impl}
					    {Enrollment Set/Partitions}
					    {datastore}{}
					\end{call}
				\end{call}

				\begin{call}{driver}
				    {\code{ReturnStatus}}{logger}{}
				\end{call}
			\end{sdblock}
		\end{sdblock}
	\end{sequencediagram}

	\captionsetup{font=footnotesize}
	\caption{Finalization of Enrollment Set.}
	\label{fig:api-finalization}
\end{figure}

\subsection{Identification: Stage One}
\label{subsec:api-identification_stage_one}

A search template and enrollment set partition number are provided to the
\implementation, which should search the enrollment set partition and record any
necessary intermediate information. The path to a $4$ GB storage area will be
provided in which this intermediate information should be written. The storage
area will be a RAM disk, and will be shared by all $N$ a identification
processes running on the same test hardware. In an effort to reduce disk I/O
from identification timing, the \testdriver will be responsible for moving data
from this RAM disk to permanent storage.

A UML sequence diagram for this process is provided in
\Cref{fig:api-identification_stage_one}.

\subsubsection{Failure to Extract}
If a search template failed to extract, it will not be passed to this method.

\subsubsection{Failure to Search}
If a failure occurs while searching during this stage, an appropriate error
code should be returned. The \implementation is \textit{still} responsible for
recording information about the search in the specified area for use in
\stagetwoidentification, if desired. All Search IDs passed to 
\stageoneidentification will be passed to \stagetwoidentification, regardless
of failure. It is the \implementation's descision whether or not to return
an empty candidate list from \stagetwoidentification as a result of a failure
in \stageoneidentification. 

\begin{figure}
	\centering
	\begin{sequencediagram}
		\newthread{driver}{\TestDriver}
		\newinst{driverchild}{\shortstack{\TestDriver\\Child}}
		\newinst{logger}{Logger}
		\newinst{datastore}{Storage}
		\newinst{interface}{\code{N2N::Interface}}
		\newinst{impl}{\shortstack{\code{N2N::Interface}\\
		    Implementation}}

		\begin{sdblock}{Identification: Stage One}{}
			\begin{call}{driver}{\code{getImplementation()}}
			    {interface}{\code{std::shared_ptr<N2N::Interface>}}
				\begin{call}{interface}{new}{impl}
				    {\code{this}}\end{call}
			\end{call}

			\begin{call}{driver}{\code{getIDs()}}{impl}{E-mail,
			    Revision, ID, \code{ReturnStatus}}
			\end{call}

			\begin{call}{driver}{E-mail, Revision, ID}{logger}{}
			\end{call}

			\begin{sdblock}{Initialization}
			    {\textit{Participant multi-threading is permitted}}
				\begin{call}{driver}{\code{initIdentification%
				    StageOne(nodeNumber, ...)}}{impl}
				    {\code{ReturnStatus}}
				\end{call}

				\begin{call}{driver}{\code{ReturnStatus}}
				    {logger}{}
				\end{call}
			\end{sdblock}

			\begin{call}{driver}{new}{driverchild}{exit}
				\begin{sdblock}{Identification Loop}
				    {\textit{$[1,n]$ processes, no participant
				    multi-threading}}
					\begin{call}{driverchild}
					    {\code{identifyTemplateStageOne(%
					    ...)}}{impl}{\code{ReturnStatus}}
						\postlevel

						\begin{callself}{impl}
						    {\shortstack{Search\\
						    Enrollment\\Set\\Partition}}
						    {}
						    \postlevel
						\end{callself}

						\begin{call}{impl}
						    {Identification Data}
						    {datastore}{}
						\end{call}
					\end{call}

					\postlevel
					\begin{call}{driverchild}
					    {\shortstack{Duration\\
					    \code{ReturnStatus}}}{logger}{}
					\end{call}
				\end{sdblock}
			\end{call}
		\end{sdblock}
	\end{sequencediagram}

	\captionsetup{font=footnotesize}
	\caption{Identification: Stage One.}
	\label{fig:api-identification_stage_one}
\end{figure}

\subsection{Identification: Stage Two}
\label{subsec:api-identitification_stage_two}

After a search template has been searched against all enrollment set partitions
in \stageoneidentification, the same search template will be provided to stage
two of identification.  During this stage, the \testdriver will provide all
information written during \stageoneidentification for this search template.
The \implementation should coalesce this information and/or perform more
searching over all partitions and return a candidate list.
\stagetwoidentification will occur in a separate process with a separate
instantiation of of the \implementation.

A UML sequence diagram for this process is provided in
\Cref{fig:api-identification_stage_two}.

\subsubsection{Candidate List}
A maximum of $100$ candidates may be returned. There is no minimum number of
candidates. The number of candidates in the candidate list will be determined
by calling the \code{size()} method of \code{vector}, and so all entries in the
candidate list must be valid candidates (not ``placeholder'' objects).
Candidates in the list must be sorted by descending similarity score, where
\code{candidate.at(0)} has the highest probability of being a match.

\subsubsection{Failure to Extract}
If a search template failed to extract, it will not be passed to this method.
The \testdriver will automatically record an empty candidate list.

\begin{figure}
	\centering
	\begin{sequencediagram}
		\newthread{driver}{\TestDriver}
		\newinst{driverchild}{\shortstack{\TestDriver\\Child}}
		\newinst{logger}{Logger}
		\newinst{datastore}{Storage}
		\newinst{interface}{\code{N2N::Interface}}
		\newinst{impl}{\shortstack{\code{N2N::Interface}\\
		    Implementation}}

		\begin{sdblock}{Identification: Stage Two}{}
			\begin{call}{driver}{\code{getImplementation()}}
			    {interface}{\code{std::shared_ptr<N2N::Interface>}}
				\begin{call}{interface}{new}{impl}
				    {\code{this}}\end{call}
			\end{call}


			\begin{call}{driver}{\code{getIDs()}}{impl}{E-mail,
			    Revision, ID, \code{ReturnStatus}}
			\end{call}

			\begin{call}{driver}{E-mail, Revision, ID}{logger}{}
			\end{call}

			\begin{sdblock}{Initialization}{\textit{Participant
			    multi-threading is permitted}}
				\begin{call}{driver}{\code{initIdentification%
				    StageTwo(...)}}{impl}{\code{ReturnStatus}}
				\end{call}

				\begin{call}{driver}{\code{ReturnStatus}}
				    {logger}{}
				\end{call}
			\end{sdblock}

			\begin{call}{driver}{new}{driverchild}{exit}
				\begin{sdblock}{Identification Loop}%
				    {\textit{$[1,n]$ processes, no participant
				    multi-threading}}
					\begin{call}{driverchild}
					    {\code{identifyTemplateStageTwo(%
					    ...)}}{impl}{\shortstack{Candidate
					    List\\\code{ReturnStatus}}}
					    	\postlevel
						\begin{callself}{impl}
						{\shortstack{Search\\
						Enrollment\\Set}}{}
						\postlevel
						\end{callself}
					\end{call}

					\postlevel
					\begin{call}{driverchild}
					    {\shortstack{Duration\\
					    \code{ReturnStatus}}}{logger}{}
					\end{call}
				\end{sdblock}
			\end{call}
		\end{sdblock}
	\end{sequencediagram}

	\captionsetup{font=footnotesize}
	\caption{Identification: Stage Two.}
	\label{fig:api-identification_stage_two}
\end{figure}

