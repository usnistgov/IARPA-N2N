\section{Image Data Interchange}
\label{sec:data-interchange}
The \testdriver is responsible for decompressing all standard imagery and
providing it to the \lib. This section defines the format in which images
will be transmitted to the \lib. The \testdriver aims to transmit images
in a standarized format to prevent participants from needing to implement new
image-handling routines.

\subsection{Scan Lines}
Images shall follow the scan sequence as defined by ISO/IEC 19794-4:2005, \S6.2,
and paraphrased here. \Cref{fig:data-scanlines} illustrates the recording order
for the scanned image. The origin is the upper left corner of the image. The
$X$-coordinate position (horizontal) shall increase positively from the origin
to the right side of the image. The $Y$-coordinate position (vertical) shall
increase positively from the origin to the bottom of the image.

\begin{figure}
	\centering
	\fbox{%
	\begin{tikzpicture}
		\node[inner sep = 0] (fingerprintImage)
		{\includegraphics[height=1.5in]{fingerprint}};

		\node[rectangle] (line0)
		    [right=0.5in of fingerprintImage.north east] {Line $0$};
		\node[rectangle] (lineK)
		    [right=0.5in of fingerprintImage] {Line $k$};
		\node[rectangle] (lineN)
		    [right=0.5in of fingerprintImage.south east] {Line $n$};

		\node[minimum width=3in, minimum height = 0.25in, draw]
		    (imageArray)
		    [right=0.5in of lineK] {};
		\node (line0Array)
		    [right=0pt of imageArray.west] {Line $0$};
		\node (lineKArray)
		    [right=-4ex of imageArray.center] {Line $k$};
		\node (lineNArray)
		    [left=0pt of imageArray.east] {Line $n$};

		\node (origin)
		    [above=0pt of fingerprintImage.north west]
		    {\footnotesize$(0,0)$};

		\draw[->, very thick, orange]
		    (fingerprintImage.north west) -- (line0.west);
		\draw[->, very thick, blue]
		    (fingerprintImage.west) -- (lineK.west);
		\draw[->, very thick, red]
		    (fingerprintImage.south west) -- (lineN.west);

		\draw[->, very thick, orange] (line0.east) -|
		    ([yshift=1pt]line0Array.north);
		\draw[->, very thick, blue] (lineK.east) -- (4.75, 0) --
		    (4.75, -1.25) -| ([yshift=-1pt]lineKArray.south);
		\draw[->, very thick, red] (lineN.east) -|
		    ([yshift=-1pt]lineNArray.south);
	\end{tikzpicture}}

	\captionsetup{font=footnotesize}
	\caption{Order of scanned lines.}
	\label{fig:data-scanlines}
\end{figure}

\subsection{Colors}
Ensure that the appropriate values are passed to your \lib
by ensuring the correct encoding of information in the JPEG-2000 header. A
sample program, \href{https://github.com/usnistgov/IARPA-N2N/blob/master/src/utilities/beImageInfo.cpp}{beImageInfo.cpp},
can be used to mimic the decompression
of images that \beframework will perform.

\subsubsection{Colorspace/Number of Channels}
RGBA (four channels), RGB (three channels), and Grayscale (one channel)
colorspaces are permitted for \libs. When using the RGB colorspace, pixels will
be passed with the red channel first, followed by the green channel, and then
the blue channel.  If an alpha channel is provided, it will immediately follow
the blue channel. All channels will have the same numbers of bits when
decompressed.

The number of bits per pixel of the image can be determined by dividing the
result of an \code{Image} instance's \code{getColorDepth()} method by the result of
\code{getBitDepth()}.

\subsection{Channel Depth}
The number of bits per channel of the image can be determined by calling the
\code{getBitDepth()} method on the \code{Image} instance. The API does not
support channel bit depths other than $8$ or $16$. If the provided \scanner
creates images in a lower bit depth, it should pad the channel values to fit
within 8 or 16 bits. Bit depths larger than 16 are not supported.

\subsubsection{8-bit}
The minimum value that will be assigned to a ``black'' pixel is \code{0x00}. The
maximum value that will be assigned to a ``white'' pixel is \code{0xFF}.
Intermediate gray levels will have assigned values of \code{0x01}--\code{0xFE}.
The pixels are stored left to right, top to bottom, with one 8-bit byte per
pixel.

The number of bytes in an image is equal to its height multiplied by its width
as measured in pixels. The image height and width in pixels will be supplied to
the software library as supplemental information.

\subsubsection{16-bit}
The minimum value that will be assigned to a ``black'' pixel is \code{0x00 00}.
The maximum value that will be assigned to a ``white'' pixel is \code{0xFF FF}.
Intermediate gray levels will have assigned values of \code{0x00 01}--\code{0xFF
FE}. The pixels are stored left to right, top to bottom, with two 8-bit bytes
per pixel. Data will be provided in little-endian format, where the most
significant bit of the channel's value will be at the 9th of 16 bits. For
example, the intermediate value $4\,660$ (\code{0x12 34}) would be transmitted
as \code{0b0011 0100 0001 0010}.

The number of bytes in an image is equal to its height multiplied by its width
as measured in pixels, multiplied by two. The image height and width in pixels
will be supplied to the software library as supplemental information.

