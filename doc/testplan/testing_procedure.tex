\section{Testing Procedure}
\label{sec:procedure}

Because of the live data collection portion of the \project, the advertised
deadlines (\Cref{subsec:intro-timeline}) are very strict and cannot be
extended. Failure to adhere to these deadlines may result in disqualification.

\subsection{Validation}

All participants submit their \lib via the project's validation mechanism. Any
\lib submitted in a different manner will be asked to resubmit using the project
validation mechanism.

\subsection{Timing Test}
Upon receiving a validated submission from a participant, a timing test will be
run. This test will confirm whether or not the submission meets the timing
thresholds defined in \Cref{subsec:software-speed}. If the submission is not
fast enough and there is time prior to the final submission deadline,
participants will be asked to speed up their algorithm and resubmit.

Note that not all timing tests may be performed. For example, finalization of
a multi-million subject enrolled set may not be feasible until after the submission
deadline. Participants should ensure that all timing metrics are met prior to
submission.

\subsection{``Legacy'' Enrollment Template Generation}
After timing has been confirmed, a large number of ``legacy'' rolled
fingerprints will be provided to the enrollment template creation method
(\Cref{subsec:api-enrollment_template}). This process is expected to be completed
for all participants \textit{prior} to the live data collection.

\subsection{Record Generation}
Once the live data collection has completed, the test facility will collect the
images and optional proprietary data from the participant's \scanners, along
with the ``legacy'' rolled captures from ``Operator A'' and ``Operator B.'' All
images will be encoded with subject information in an ANSI/NIST-ITL file.

\subsection{N2N Enrollment Template Generation}
The Operator B and participant \scanner images extracted from the ANSI/NIST-ITL
files provided by the test facility will be passed to
\code{makeEnrollmentTemplate()}.

\subsection{Enrollment Set Finalization}
At this point, all enrollment templates will have been created. Two enrollment
sets will be formed by two separate calls to the enrollment set finalization
method(\Cref{subsec:api-finalization}). One will be a combination of the
``legacy'' and ``Operator B'' imagery, and the other a combination of the
``legacy'' and participant \scanner imagery.

\subsection{Search Template Creation}
While simultaneously finishing enrollment template creation and enrollment set
finalization, search template creation (\Cref{subsec:api-search_template}) will
commence in order to create several search templates. It will be necessary to
create search templates from ``Operator A'' and latent imagery.

\subsection{Two-Stage Identification}
After all enrollment sets have been finalized and all search templates have been
created, the two-stage identification process will begin.

\subsection{Analysis and Results Notification}
Candidate lists and timing will be analyzed after the completion of the
two-stage identification. The project sponsor will be notified of the results
and will notify the recipients. Only the project sponsor will be able to release
results.

\subsection{Supplemental Report Generation}
Data beyond which participant was the fastest or most accurate will be
generated from running the software portion of this \project. Supplemental
reports detailing this information may be released at the discretion of the
project sponsor.
