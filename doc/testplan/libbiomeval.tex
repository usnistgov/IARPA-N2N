\section{Biometric Evaluation Framework}
\label{sec:libbiomeval}

Classes from NIST's \textit{Biometric Evaluation Framework} will be used when
writing software to the \project API.  These classes are designed to be easy to
learn and use. Methods and concepts that will be necessary during implementation
are detailed in this section.

Participants are free to use other parts of
\textit{Biometric Evaluation Framework}
in their implementations. There are two restrictions necessary to ensure that
the \testdriver, which relies on \textit{Biometric Evaluation Framework},
continues to work correctly. Participants shall not modify the source code of
the library itself (though if you find a bug, please feel free to
\href{https://github.com/usnistgov/BiometricEvaluation/issues}{submit an
issue}). Participants shall also not import identical symbol definitions into
their library. The \testdriver will be linked against \libbiomevalVersion, which
will include all symbols necessary for participant \libs.

\subsection{BiometricEvaluation::Finger}

\large{\code{/usr/local/include/be_finger.h}}

The \code{Finger} namespace contains enumerated constants that will be used to
transmit information about the fingers represented in transmitted imagery.

Notable enumerations include:
\begin{itemize}
	\item \code{Finger::Position}: Finger positions, such as
	\code{RightThumb} and \code{LeftIndex}. The underlying numeric values
	correspond to the finger position codes (FGP) from ANSI/NIST-ITL.

	\item \code{Finger::Impression}: Impression type of the captured
	fingerprint.
		\begin{itemize}
			\item \textbf{NOTE}: For this test, all N2N images will
			be represented by \code{LiveScanRolled}. There will be
			no differentiation between ``legacy'' rolled imagery
			and the participant's \scanner imagery.
			
			\item \textbf{NOTE}: \implementations will be notified
			during initialization whether to prepare for latent
			imagery. It is expected that a different algorithm or
			configuration within the core \lib will be needed when
			processing latent imagery.
		\end{itemize}
\end{itemize}

\subsection{BiometricEvaluation::Image}

\large{\code{/usr/local/include/be_image.h}}\\
\large{\code{/usr/local/include/be_image_image.h}}

The \code{Image::Image} class encapsulates information about standardized images
through classes defined in the \code{Image} namespace. Images provided to the
\implementation will be decompressed and encapsulated into an \code{Image::Raw}.

Notable methods include:
\begin{itemize}
	\item \code{getColorDepth()}: Obtain the number of bits per pixel.
	\item \code{getBitDepth()}: Obtain the number of bits per channel.
	\item \code{getResolution()}: Obtain the scanning resolution.
		\begin{itemize}
			\item The values are stored in \code{xRes} and
			\code{yRes} of the resulting object.
		\end{itemize}
	\item \code{getDimensions()}: Obtain the image dimensions, in pixels.
		\begin{itemize}
			\item The values are stored in \code{xSize} and
			\code{ySize} of the resulting object.
		\end{itemize}
	\item \code{getRawData()}: Obtain a buffer of decompressed raw bytes.
	\begin{itemize}
		\item \textbf{NOTE}: For \code{Image::Raw}, \code{getData()}
		will return the same information as \code{getRawData()}.
	\end{itemize}
\end{itemize}

\subsection{BiometricEvaluation::IO::RecordStore}

\large{\code{/usr/local/include/be_io_recordstore.h}}

When creating the finalized enrollment set (\Cref{subsec:api-finalization}),
enrollment templates will be provided to the \implementation through an
\code{IO::RecordStore}. These objects are key/data-pair storage abstractions
that are designed for high-performance on the test hardware filesystems.

Individual key/data-pair elements of an \code{RecordStore} are stored in an
\code{IO::Record} struct, with \code{key} and \code{data} members.
Dereferencing an \code{IO::RecordStores}'s iterator will provide an
\code{IO::Record}.

Notable methods include:
\begin{itemize}
	\item \code{getCount()}: Obtain the number of key/data pairs.
	\item \code{sequence()}: Obtain the ``next'' key/data pair, until
	exhausted.
	\item \code{begin()/end()}: Iterate through all key/data pairs.
\end{itemize}

\subsection{BiometricEvaluation::Error::Exception}

\large{\code{/usr/local/include/be_error_exception.h}}

Should any truly exceptional conditions happen in your \implementation, throw
an instance of an \code{Error::Exception}. An optional \code{std::string} can
be included with additional information.

\implementations should not throw an exception as a substitute for returning a
failure condition. Exceptions should only be thrown in situations where a
non-recoverable error has occurred.

\subsection{BiometricEvaluation::Memory::uint8Array}

\large{\code{/usr/local/include/be_memory_autoarray.h}}

A \code{Memory::uint8Array} is a specialization of \code{Memory::AutoArray} for
\code{uint8_t}. It is used in many places in the API and \code{Biometric
Evaluation Framework} for managing dynamic memory buffers. It provides the
convenience and C compatibility of a \code{uint8_t*} with the benefits of
\code{std::vector<uint8_t>}.

\code{uint8Array}s are \textit{implicitly} convertible to \code{uint8_t*} and
can be used anywhere a \code{uint8_t*} is required, such as \code{std::memcpy}.

Notable methods include:
\begin{itemize}
	\item \code{resize()}: Grow or shrink the array.
	\item \code{size()}: Obtain the number of bytes in the array.
	\item \code{operator[]}: Read or write directly from an offset without
	bounds checking.
	\item \code{begin()/end()}: Iterate through all elements.
\end{itemize}
